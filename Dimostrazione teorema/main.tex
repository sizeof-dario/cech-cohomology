\documentclass{article}
\usepackage{amsmath, amsthm, amssymb}

% ambiente teorema
\newtheorem{teorema}{Teorema}
\newtheorem{lemma}{Lemma}
\newtheorem{exercise}{Esercizio}

\begin{document}

Il teorema di riferimento sarà il seguente:
\begin{teorema}[Corollario 6.49]
    Siano $\mathcal{A}$ e $\mathcal{B}$ due categorie abeliane di cui la prima con abbastanza iniettivi e siano $(F^n,G^n:\mathcal{A}\rightarrow\mathcal{B})_{n\in\mathbb{N}_0}$ due successioni di funtori covarianti additivi tra esse. Si supponga che:
    \begin{enumerate}
        \item Per ogni successione esatta corta $0\longrightarrow A\longrightarrow B\longrightarrow C\longrightarrow 0$, esiste una successione esatta lunga dotata di omomorfismi naturali di collegamento;
        \item  $F^0$ è naturalmente isomorfo a $G^0$;
        \item $F^n(E)=0=G^n(E)$ per tutti gli oggetti iniettivi $E$ e per ogni $n\geq 1$.
    \end{enumerate}
    Allora $F^n$ è naturalmente isomorfo a $G^n$ per ogni $n\in\mathbb{N}_0$.
\end{teorema}

Per il punto 1 abbiamo (per Čech):
\begin{teorema}[Serre, teorema 6.87]
    Sia $0\longrightarrow\mathcal{F}'\longrightarrow\mathcal{F}\longrightarrow\mathcal{F}''\longrightarrow0$ una successione esatta corta di fasci di gruppi abeliani su uno spazio paracompatto. Allora esiste una successione esatta nella coomologia di Čech:
    \[
\begin{aligned}
0 \to \check{H}^0(\mathcal{F}') &
   \to \check{H}^0(\mathcal{F}) 
   \to \check{H}^0(\mathcal{F}'') \\
  &\to \check{H}^1(\mathcal{F}') 
   \to \check{H}^1(\mathcal{F}) 
   \to \check{H}^1(\mathcal{F}'') 
   \to \cdots \\
  &\to \check{H}^q(\mathcal{F}') 
   \to \check{H}^q(\mathcal{F}) 
   \to \check{H}^q(\mathcal{F}'') 
   \to \check{H}^{q+1}(\mathcal{F}') 
   \to \cdots \,.
\end{aligned}
\]
\end{teorema}
Per il punto 2 abbiamo (per Čech):
\begin{exercise}[6.79(i)]
    Siano $X$ uno spazio topologico, $\mathcal{U}$ un suo ricoprimento aperto e $\mathcal{F}$ un fascio di gruppi abeliani su $X$. Allora:
    $$\boxed{\check{H}^0(\mathcal{U},\mathcal{F})=\Gamma(\mathcal{F})=\mathcal{F}(X)}$$
    (a cui è uguale anche $H^0(\mathcal{F})$).
\end{exercise}

Per il punto 3 abbiamo (per Čech):
\begin{lemma}[6.85]
    Sia $X$ uno spazio topologico e sia $\mathcal{F}$ un fascio di gruppi abeliani su $X$. Se $\mathcal{F}$ è iniettivo, allora, per ogni $q\geq 1$:
    $$\check{H}^q(\mathcal{F})=\{0\}$$
\end{lemma}

\begin{teorema}
Siano $X$ uno spazio topologico e $\mathcal{F}$ un fascio di gruppi abeliani su $X$. Se $X$ è paracompatto, allora, per ogni $q\in\mathbb{N}_0$:
$$\check{H}^q(\mathcal{F})\cong H^q(\mathcal{F})$$
\end{teorema}

\end{document}