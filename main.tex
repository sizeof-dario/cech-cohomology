\documentclass{beamer}
%Information to be included in the title page:

\usetheme{Warsaw}

\usefonttheme[onlymath]{serif}

\title{Coomologia di Čech}

\author{Dario Di Meo, D70000023}


\date{Esame di Geometria Algebrica, XX/12/25}

\begin{document}

\frame{\titlepage}

\begin{frame}
\frametitle{Definizioni 1}

\begin{block}{Definizione (Complesso simpliciale astratto)}
Siano un insieme $\mathrm{Vert}(K)\ne\emptyset$ e una famiglia di suoi sottoinsiemi $\mathcal{S}_K\subseteq\mathcal{P}(\mathrm{Vert}(K))\setminus\{\emptyset\}$ tali che
\begin{itemize}
    \item $\{v\}\in\mathcal{S}_K$ per ogni $v\in\mathrm{Vert}(K)$;
    \item $\sigma\in\mathcal{S}_K\wedge\emptyset\ne\tau\subseteq\sigma\Rightarrow\tau\in\mathcal{S}_K$.
\end{itemize}
Allora $K=(\mathrm{Vert}(K),\mathcal{S}_K)$ si dice \alert{complesso simpliciale astratto}, gli elementi di $\mathrm{Vert}(K)$ si dicono \alert{vertici} e gli elementi di $\mathcal{S}$ si dicono \alert{simplessi}. 
\end{block}

Un simplesso $\sigma$ tale che $|\sigma|=n+1$ si dice $n$-simplesso.

Se $K$ e $L$ sono complessi simpliciali, una funzione $\varphi:\mathrm{Vert}(K)\rightarrow\mathrm{Vert}(L)$ tale che $\sigma\in\mathcal{S}_K\Rightarrow\varphi(\sigma)\in\mathcal{S}_L$ per ogni $\sigma\in\mathcal{S}_K$ si dice \alert{mappa simpliciale}.
\end{frame}

\begin{frame}
\frametitle{Definizioni 2}

\begin{block}{Definizione (Nerbo)}
    Siano $X$ uno spazio topologico e $\mathcal{U}=(U_I)_{i\in I}$ un suo ricoprimento aperto. Si chiama \alert{nerbo} il complesso simpliciale astratto $N(\mathcal{U})$ di vertici $\mathrm{Vert}(\mathcal{U})=\mathcal{U}$ e simplessi le sottofamiglie $\{U_{i_0},U_{i_1}\ldots,U_{i_n}\}\subseteq\mathcal{U}$ tali che $\bigcap_{j=0}^nU_{n_j}\ne\emptyset$, al variare di $n\in\mathbb{N}_0$.
\end{block}


\end{frame}

\end{document}